\documentclass[12pt,oneside,a4paper]{article}

%% Language and font encodings

\usepackage[english]{babel}
\usepackage[utf8x]{inputenc}
\usepackage[T1]{fontenc}
\usepackage{amssymb,tikz,pdftexcmds,xparse}


%change caption of contents e appendices

\addto\captionsenglish{% Replace "english" with the language you use
	\renewcommand{\figurename}{Figura}%
	\renewcommand{\contentsname}{Table of Contents}%
	\renewcommand{\contentsname}{Contenuti}
}


%% Give you the possibiliy of split text
\usepackage{seqsplit}

%% Sets page size and margins
\usepackage[a4paper,top=3cm,bottom=2cm,left=3cm,right=3cm,marginparwidth=1.75cm]{geometry}

%%allow to break url
\usepackage[hyphens]{url}

%% Useful packages
\usepackage{amsmath}
\usepackage{graphicx}
\graphicspath{ {./immagini/} }
\usepackage[colorinlistoftodos]{todonotes}
\usepackage[colorlinks=true, allcolors=black]{hyperref}
\usepackage{todonotes}
\usepackage{multicol}

%appendice 
\usepackage[toc,page]{appendix}
\renewcommand\appendixtocname{Appendice}
\renewcommand\appendixpagename{Appendice}
%% Personal function

%no label
\usepackage[nolabel, final]{showlabels}

%% Solidity env, per ora è java
\usepackage{listings}
\usepackage{color}

\definecolor{dkgreen}{rgb}{0,0.6,0}
\definecolor{gray}{rgb}{0.5,0.5,0.5}
\definecolor{mauve}{rgb}{0.58,0,0.82}

\lstset{frame=tb,
	language=Java,
	aboveskip=3mm,
	belowskip=3mm,
	showstringspaces=false,
	columns=flexible,
	basicstyle={\small\ttfamily},
	numbers=none,
	numberstyle=\tiny\color{gray},
	keywordstyle=\color{blue},
	commentstyle=\color{dkgreen},
	stringstyle=\color{mauve},
	breaklines=true,
	breakatwhitespace=true,
	tabsize=3
}


	\title{Movimento di un robot 2 motori rear e una sfera front}
	\author{Luca Vecchi - STUFFCUBE}
	\date{20 Aprile 2017}

\begin{document}
	%\input{./Struttura/Frontespizio.tex}
	

	\maketitle
	\clearpage



	%	\input{Struttura/Introduzione.tex}
	
	\section{Curvatura}
	
	Vogliamo astrarre il movimento di Arianna in due tipologie diverse: rette e circonferenze. Una futura implementazione delle curve bezier (manca di trovare come fare l'offset e di modificare il path) potrà dare una completa gestione dei movimenti potendo controllare anche l'angolo di arrivo al punto.
	
	Possiamo considerare il contenuto informativo di $S_0 \bigcup r_0$ al pari di $\alpha \bigcup r_0$ e $S_0 \bigcup \alpha$ per via delle ovvie uguaglianze.
	Nel nostro caso per le rette andremo a definire la funzione \textit{Retta} come segue:
	retta ($S_0$): $S_s , S_d$
	
	, mentre nel caso delle circonferenze:
	
	circ($\alpha$,$r_o$,$\Delta$):  $S_s , S_d$
	
	dove $\delta$ è la distanza tra il centro del robot e la circonferenza tracciata dal movimento del centro delle ruote.
	
	\begin{equation*}
	S_0 \equiv \alpha * 2 * \pi * r_0 \div 360
	\end{equation*}
	\begin{equation*}
	S_1 \equiv \alpha * 2 * \pi * (r_0 - \Delta) \div 360
	\end{equation*}	
	\begin{equation*}
	S_2 \equiv \alpha * 2 * \pi * (r_0 + \Delta) \div 360
	\end{equation*}
	
	Come si evince dall'immagine \ref{fig:circonferenze} si può notare le circonferenze concentriche, rappresentate dalla più interna verso l'esterna, come il tragitto effettuato dalla ruota più interna, dal centro del robot e dalla ruota più esterna.
	
	La difficoltà sta nella gestione dei motori in modo sincrono in quanto perchè il centro del robot effettui una circonferenza perfetta, la velocità delle ruote deve essere mantenuta costante\footnote{Vedi libreria Accelstepper}.
	
	
	\clearpage


	%% Bibliography .bib
	%\bibliography{./Struttura/bib.bib}
	%\bibliographystyle{unsrt}	
\end{document}
